\documentclass{report}

\input{preamble}
\input{macros}
\input{letterfonts}
\usepackage{float}

\title{\Huge{Perception and AI Skin detection}\\ ESIEE Paris 2023 - 2024}
\author{\huge{Gloria Desideri and Julie  Aguilar}}
\date{}

\begin{document}

\maketitle
\newpage% or \cleardoublepage
% \pdfbookmark[<level>]{<title>}{<dest>}
\pdfbookmark[section]{\contentsname}{toc}
\tableofcontents
\pagebreak

\chapter{Introduction}
The main objective of this practical session was to wxperiment with color manipulation
to understand the impact of color coding and light conditions. We had to perform
skin segmentation from images of hands in three different light conditions: normail lighting,
down lighting and upligthing. We were aslso asked to use two different color spaces. For 
the purpose of this lab we choosed the HSV and the YCrCb spaces.\\
The YCbCr color space, also known as YUV, is a color encoding system often used in video and image compression. It separates the chrominance (color information) from the luma (brightness information) to allow for greater compression efficiency.
\begin{itemize}
  \item The Y represents the luma or brightness of the image
  \item Cb represents the difference between the blue and green components of the original color
  \item Cr component represents the difference between the red and green components of the original color
\end{itemize}
The HSV (Hue, Saturation, Value) color space is a model that represents colors in a way that's closer to how humans perceive them. Unlike the RGB (Red, Green, Blue) color model, which uses primary colors, HSV separates the color information into three components:
\begin{itemize}
  \item Hue: This represents the type of color. It is usually represented in degrees from 0 to 360, where 0 (or 360) is red, 120 is green, and 240 is blue. Other colors fall within these ranges. 
  \item Saturation: This measures the purity of the color. A saturation of 0 represents a shade of gray, while a saturation of 1 represents the full color.
  \item Value: This represents the brightness of the color. A value of 0 represents black, a value of 1 represents the brightest version of the color, and values in between represent various shades of the color.
\end{itemize}
\begin{figure}[H]
\includegraphics[scale=0.3]{Images/hsv.png}
\centering
\end{figure}
\chapter{Data}
The assignment asked for a few images in three different lighting conditions. We sampled 
four images from the 11k hands dataset.
\begin{figure}[H]
  \minipage{0.25\textwidth}
    \includegraphics[width=\linewidth]{Images/hand1.png}
    \caption{Hand 1}
  \endminipage\hfill
  \minipage{0.25\textwidth}
    \includegraphics[width=\linewidth]{Images/hand2.png}
    \caption{Hand 2}
  \endminipage\hfill
  \minipage{0.25\textwidth}
    \includegraphics[width=\linewidth]{Images/hand3.png}
    \caption{Hand 3}
  \endminipage\hfill
  \minipage{0.25\textwidth}
    \includegraphics[width=\linewidth]{Images/hand4.png}
    \caption{Hand 4}
  \endminipage\hfill
\end{figure}
Using an online tool we also got the same images without 
background which allowed us to automatically extract a binary groud truth.
In order to obtain three different lighting conditions we used the PIL python package enhancer,
setting the factor to 0,5 for down lighting and to 1,5 for up lighting. 
\begin{figure}[H]
  \minipage{0.32\textwidth}
    \includegraphics[width=\linewidth]{Images/hand_down.png}
    \caption{Hand down}
  \endminipage\hfill
  \minipage{0.32\textwidth}
    \includegraphics[width=\linewidth]{Images/hand_normal.png}
    \caption{Hand normal}
  \endminipage\hfill
  \minipage{0.32\textwidth}
    \includegraphics[width=\linewidth]{Images/hand_up.png}
    \caption{Hand up}
  \endminipage\hfill
\end{figure}
\chapter{Analysis}
The whole process of segmentation and analysis was carried out using the OpenCV python package
for both color spaces. Before proceding with the segmentation we performed a histogram analysis
which gave us some insights for the tresholds of the segmentation.
\section{YCrCb space}
The first space we analyzed was the YCrCb and below you can see the histograms for 
the three values.
\begin{figure}[H]
\includegraphics[scale=0.4]{Images/YCRCBNormal.png}
 \caption{Normal image}
\centering
\end{figure}
\begin{figure}[H]
\includegraphics[scale=0.4]{Images/YCRCBUp.png}
 \caption{Up lighting image}
\centering
\end{figure}
\begin{figure}[H]
\includegraphics[scale=0.4]{Images/YCRCBDown.png}
 \caption{Down lighting image}
\centering
\end{figure}
Here is evident how the image with poor lighting conditions is less bright (red line) while the up lighted
image has an higher brightness, also the two channels representing the differences are accentuated
with poor and normal lighting.
\section{HSV}
We proceded our analysis with the HSV space
\begin{figure}[H]
\includegraphics[scale=0.4]{Images/HSVNormal.png}
 \caption{Normal image}
\centering
\end{figure}
\begin{figure}[H]
\includegraphics[scale=0.4]{Images/HSVUp.png}
 \caption{Up lighting image}
\centering
\end{figure}
\begin{figure}[H]
\includegraphics[scale=0.4]{Images/HSVDown.png}
 \caption{Down lighting image}
\centering
\end{figure}
Here we can see how the image with great brightness has the higher S and V values. 
Also the histograms give much insight for the hue value, which is from 0 to 100 for normal and
poor lighting and more restricted for up lighting. \\
In addition we analyzed the HSV graphs with some images we took with our cameras and they produced the following results.
\begin{figure}[H]
\includegraphics[scale=0.4]{Images/HSVNormal_J.png}
 \caption{Image taken without flash}
\centering
\end{figure}
\begin{figure}[H]
\includegraphics[scale=0.4]{Images/HSVUp_J.png}
  \caption{Image taken with flash}
\centering
\end{figure}
\begin{figure}[H]
\includegraphics[scale=0.4]{Images/HSVDown_J.png}
 \caption{Poor lighting}
\centering
\end{figure}
The analysis of HSV graphs for skin color detection provides crucial insights into the method's performance under different lighting conditions. In optimal conditions with a flash, a distinct peak is observed from 0 to 12.5, suggesting effective detection of skin tones under bright light, followed by a peak from 12.5 to 25 with a slight decrease in brightness. The subsequent stability with peaks at 0.005 may indicate areas with less light or variations in skin color.

However, without a flash, the peak from 9 to 12.5 suggests a slight variation in skin color, while the peak from 25 to 37.5 represents another range of skin colors with decreased brightness. The subsequent stability with a peak at 0.0155 may correspond to stable but less bright areas. In very low-light conditions, although the method still detects a range of skin colors from 12.5 to 25, peaks at 0.0075 towards 37.5, 12.5, and 150 may indicate detection errors or variations due to low light, with peaks abruptly stopping at the value of 162.5.
\chapter{Segmentation and results}
\section{First tests}
As a start we used the method proposed in class on the images taken with our camera. Since this was
a mere test we only conducted this experiment on the HSV space obtaining the following results.
\begin{figure}[H]
  \minipage{0.32\textwidth}
    \includegraphics[width=\linewidth]{Images/hand_downHSV_J.png}
    \caption{Hand down}
  \endminipage\hfill
  \minipage{0.32\textwidth}
    \includegraphics[width=\linewidth]{Images/hand_normalHSV_J.png}
    \caption{Hand normal}
  \endminipage\hfill
  \minipage{0.32\textwidth}
    \includegraphics[width=\linewidth]{Images/hand_upHSV_J.png}
    \caption{Hand up}
  \endminipage\hfill
\end{figure}
In terms of overall performance, the method appears robust in optimal lighting conditions, with peaks corresponding to different skin tones. However, weaknesses emerge in low-light conditions, where peaks unrelated to skin color may occur. Sensitivity to lighting variations emphasizes the need for adaptation and manual validations. The method might show vulnerabilities in suboptimal conditions, requiring threshold adjustments and thorough validation to ensure reliable performance. Adaptability to diverse skin tones and manual ground truth creation will be crucial for accurate method evaluation.
\section{More formal experiments}
Next we proceded with some formal experiments.
In particular we used the precision function as
our performance measure 
\[
  \frac{\text{TP}}{\text{TP}+\text{FP}}
\]
Namely we performed a summation over the result of the logical AND between the identified region
and the ground truth then we devided the result for the summation over the identified region.
Both the identified region and the ground truth are binary images. \\
We used different parameters according to the space and light condition. In particular for the
YCrCb space we used:
\begin{itemize}
  \item [0,133, 77],[255, 173, 127] for normal lighting
  \item [0,129, 77],[150, 150, 127] for normal lighting
  \item [0,120, 77],[255, 190, 127] for normal lighting
\end{itemize}
For which we obtained a mean precision of 0.96047 for normal lighting 0.96509 for brighter images
and 0.92717 for poor lighting.\\
And for which we report the following sample results
\begin{figure}[H]
  \minipage{0.32\textwidth}
    \includegraphics[width=\linewidth]{Images/hand_downYCRCB.png}
    \caption{Hand down}
  \endminipage\hfill
  \minipage{0.32\textwidth}
    \includegraphics[width=\linewidth]{Images/hand_normalYCRCB.png}
    \caption{Hand normal}
  \endminipage\hfill
  \minipage{0.32\textwidth}
    \includegraphics[width=\linewidth]{Images/hand_upYCRCB.png}
    \caption{Hand up}
  \endminipage\hfill
\end{figure}
For HSV we used:
\begin{itemize}
  \item [0,50, 0],[200, 255, 255] for normal lighting
  \item  [0,10, 150],[100, 255, 255] for normal lighting
  \item [0,50, 0],[200, 255, 255]  for normal lighting
\end{itemize}
For which we obtained a mean precision of 0.98234 for normal lighting 0.96509 for brighter images
and 0.982198 for poor lighting.\\
And for which we report the following sample results\\
\begin{figure}[H]
  \minipage{0.32\textwidth}
    \includegraphics[width=\linewidth]{Images/hand_downHSV.png}
    \caption{Hand down}
  \endminipage\hfill
  \minipage{0.32\textwidth}
    \includegraphics[width=\linewidth]{Images/hand_normalHSV.png}
    \caption{Hand normal}
  \endminipage\hfill
  \minipage{0.32\textwidth}
    \includegraphics[width=\linewidth]{Images/hand_upHSV.png}
    \caption{Hand up}
  \endminipage\hfill
\end{figure}
\end{document}
